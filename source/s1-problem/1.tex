\documentclass[11pt]{article}
\usepackage[numbers]{natbib}
\PassOptionsToPackage{dvipdfmx}{graphicx}
\usepackage{graphicx}
\usepackage{amsmath}
\usepackage{amsfonts}
\usepackage{amssymb}
\usepackage{color}
\usepackage{verbatim}
\usepackage{fontawesome5}
\usepackage{amsthm}
\usepackage{tikz}
\usepackage{tikz-3dplot}
\usepackage{mathrsfs}
\usepackage{float}
\usepackage{enumerate}
\usepackage{ascmac}
\usepackage{fancyhdr}
\usepackage{babel}
\usepackage{fontspec}
\usepackage{xeCJK}
\usepackage{xeCJK-listings}
\usepackage{xeCJKfntef}
\usepackage{zxjatype}
\usepackage[a4paper, left=1in, right=1in, top=1in, bottom=1in]{geometry}
\usepackage{microtype}
\setjamainfont{ipaexm.ttf}
\emergencystretch=2em
\newfontfamily\CJKfont{ipaexm.ttf}
\newfontfamily\CJKchinese{宋体}
\newtheorem*{df}{Def}
\newtheorem*{thm}{Thm}
\newtheorem*{prop}{Prop}
\newtheorem*{lem}{Lem}
\newtheorem*{cor}{Cor}
\newtheorem*{rem}{Rem}
\newtheorem*{ex}{e.g.}
\newtheorem*{prob}{Problem}
\fancypagestyle{plain}{\fancyhf{} \fancyhead[R]{\CJKfont{学籍番号: 1124076 氏名:} {\CJKfontspec{宋体}程铪炘}} \cfoot{\thepage}\renewcommand{\headrulewidth}{0.4pt}}
\pagestyle{plain}
\renewcommand{\refname}{\CJKfont{参考文献}}
\newcommand{\ds}{\displaystyle}
\begin{document}
	\CJKfont
	\begin{center}
		$\S1$
	\end{center}
	\section*{(1)}
	$A_{1}\subset A_{2}\subset\cdots$より
	\begin{align*}
		P\left(\bigcup\limits_{k=1}^{\infty}A_{k}\right)
		&=P\left(A_{1}\cup\left(A_{2}-A_{1}\right)\cup\left(A_{3}-A_{2}\right)\cup\cdots\right)\\
		&=P\left(A_{1}\right)+\ds\sum_{k=2}^{\infty}P\left(A_{k}-A_{k-1}\right)
	\end{align*}
	言い換えれば、$n$番目までの和集合の確率は$A_{n}$である、よって
	\begin{align*}
		P\left(\bigcup\limits_{k=1}^{\infty}A_{k}\right)&=\ds\lim_{k\rightarrow\infty}P\left(A_{k}\right)
	\end{align*}
	
	\section*{(2)}
	$B_{k}=\Omega-A_{k}$とすると、$B_{1}\subset B_{2}\subset\cdots$から、(1)を利用すると
	\begin{align*}
		P\left(\Omega-\bigcap\limits_{k=1}^{\infty}A_{k}\right)
		&=P\left(\bigcup\limits_{k=1}^{\infty}B_{k}\right)\\
		1-P\left(\bigcap\limits_{k=1}^{\infty}A_{k}\right)&=P\left(\bigcup\limits_{k=1}^{\infty}B_{k}\right)\\
		1-P\left(\bigcap\limits_{k=1}^{\infty}A_{k}\right)&=\ds\lim_{k\rightarrow\infty}P\left(B_{k}\right)\\
		1-P\left(\bigcap\limits_{k=1}^{\infty}A_{k}\right)&=\ds\lim_{k\rightarrow\infty}P\left(\Omega-A_{k}\right)\\
		1-P\left(\bigcap\limits_{k=1}^{\infty}A_{k}\right)&=1-\ds\lim_{k\rightarrow\infty}P\left(A_{k}\right)\\
		P\left(\bigcap\limits_{k=1}^{\infty}A_{k}\right)&=\ds\lim_{k\rightarrow\infty}P\left(A_{k}\right)
	\end{align*}
	
	\section*{(3)}
	\begin{align*}
		\bigcap\limits_{n=1}^{\infty}\bigcup\limits_{k=n}^{\infty}A_{k}
		&=\left(\bigcup\limits_{k=1}^{\infty}A_{k}\right)\cap\left(\bigcup\limits_{k=2}^{\infty}A_{k}\right)\cap\left(\bigcup\limits_{k=3}^{\infty}A_{k}\right)\cap\cdots
	\end{align*}
	ここで、$\forall k\in\mathbb{N},A_{k}\in\mathscr{F}$であるから、$\forall n\in\mathbb{N},\bigcup\limits_{k=n}^{\infty}A_{k}\in\mathscr{F}$\\
	また、各$n$に対し、$A_{k^{n}}:=\bigcup\limits_{k=n}^{\infty}A_{k}$で書くと\\
	$A_{k^{1}},A_{k^{2}},\cdots,A_{k^{n}},\cdots\in\mathscr{F}$から、$\bigcap\limits_{n=1}^{\infty}A_{k^{n}}\in\mathscr{F}$\\
	言い換えれば、$\bigcap\limits_{n=1}^{\infty}\bigcup\limits_{k=n}^{\infty}A_{k}\in\mathscr{F}$
	
	\section*{(4)}
	ここで$\bigcap\limits_{n=1}^{\infty}\bigcup\limits_{k=n}^{\infty}A_{k}\in\mathscr{F}$という存在性はもう証明したから、以下は計算だけ
	\begin{align*}
		P\left(\bigcap\limits_{n=1}^{\infty}\bigcup\limits_{k=n}^{\infty}A_{k}\right)
		&=\ds\lim_{n\rightarrow\infty}P\left(\bigcup\limits_{k=n}^{\infty}A_{k}\right)\\
		&\leq\ds\lim_{n\rightarrow\infty}\ds\sum_{k=n}^{\infty}P\left(A_{k}\right)\\
		&=0
	\end{align*}
	
	
	
	
	
	
	
	
	
	
	
	
	
	
	
	
	
	
	
	
	
	
	
	
	
	
	
	
	
	
	
	
	
	
	
	
	
	
	\newpage
	\bibliographystyle{plain}
	\bibliography{references}
\end{document}
