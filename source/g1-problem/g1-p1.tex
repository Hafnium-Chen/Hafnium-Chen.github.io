\documentclass[11pt]{article}
\usepackage[numbers]{natbib}
\PassOptionsToPackage{dvipdfmx}{graphicx}
\usepackage{graphicx}
\usepackage{amsmath}
\usepackage{amsfonts}
\usepackage{amssymb}
\usepackage{color}
\usepackage{verbatim}
\usepackage{fontawesome5}
\usepackage{amsthm}
\usepackage{tikz}
\usepackage{tikz-3dplot}
\usepackage{float}
\usepackage{enumerate}
\usepackage{ascmac}
\usepackage{fancyhdr}
\usepackage{babel}
\usepackage{fontspec}
\usepackage{xeCJK}
\usepackage{xeCJK-listings}
\usepackage{xeCJKfntef}
\usepackage{zxjatype}
\usepackage[a4paper, left=1in, right=1in, top=1in, bottom=1in]{geometry}
\usepackage{microtype}
\setjamainfont{ipaexm.ttf}
\emergencystretch=2em
\newfontfamily\CJKfont{ipaexm.ttf}
\newfontfamily\CJKchinese{宋体}
\newtheorem*{df}{Def}
\newtheorem*{thm}{Thm}
\newtheorem*{prop}{Prop}
\newtheorem*{lem}{Lem}
\newtheorem*{cor}{Cor}
\newtheorem*{rem}{Rem}
\newtheorem*{ex}{e.g.}
\newtheorem*{prob}{Problem}
\fancypagestyle{plain}{\fancyhf{} \fancyhead[R]{\CJKfont{学籍番号: 1124076 氏名:} {\CJKfontspec{宋体}程铪炘}} \cfoot{\thepage}\renewcommand{\headrulewidth}{0.4pt}}
\pagestyle{plain}
\renewcommand{\refname}{\CJKfont{参考文献}}
\newcommand{\ds}{\displaystyle}
\begin{document}
	\CJKfont
	\begin{center}
		$\S 1$
	\end{center}
	\section*{1.1}
	\begin{align*}
		\ds\int_{C}\left(2x-y\right)\mathrm{d}x+\left(x+y\right)\mathrm{d}y
		&=\ds\int_{0}^{1}\left(2t-t^{2}\right)\mathrm{d}t+\left(t+t^{2}\right)2t\mathrm{d}t\\
		&=\ds\int_{0}^{1}\left(2t^{3}+t^{2}+2t\right)\mathrm{d}t\\
		&=\left[\dfrac{1}{2}t^{4}+\dfrac{1}{3}t^{3}+t^{2}\right]_{0}^{1}\\
		&=\dfrac{11}{6}
	\end{align*}
	
	\section*{1.2}
	\begin{align*}
		\ds\int_{C}\mathbf{v}\cdot\mathrm{d}\mathbf{x}
		&=\ds\int_{0}^{1}\left(\begin{array}{ccc}
			2\left(a+t\right)\left(b+t\right)c\\
			\left(a+t\right)^{2}c\\
			\left(a+t\right)^{2}\left(b+t\right)
		\end{array}\right)\cdot\left(\begin{array}{ccc}
		    1\\1\\0
		\end{array}\right)\mathrm{d}t+\ds\int_{1}^{2}\left(\begin{array}{ccc}
		    2\left(a+1\right)\left(b+1\right)\left(c-1+t\right)\\
		    \left(a+1\right)^{2}\left(c-1+t\right)\\
		    \left(a+1\right)^{2}\left(b+1\right)
		\end{array}\right)\cdot\left(\begin{array}{ccc}
		    0\\0\\1
		\end{array}\right)\mathrm{d}t\\
		&=\ds\int_{0}^{1}\left(2\left(a+t\right)\left(b+t\right)c+\left(a+t\right)^{2}c\right)\mathrm{d}t+\ds\int_{1}^{2}\left(a+1\right)^{2}\left(b+1\right)\mathrm{d}t\\
		&=c\left(a^{2}+2a\left(b+1\right)+b+1\right)+\left(a+1\right)^{2}\left(b+1\right)\\
		&=a^{2}\left(b+c+1\right)+\left(2a+1\right)\left(b+1\right)\left(c+1\right)
	\end{align*}
	また、線積分の性質を考えると
	\begin{align*}
		\ds\int_{C}\mathbf{v}\cdot\mathrm{d}\mathbf{x}
		&=\ds\int_{C}\nabla\left(x^{2}yz\right)\mathrm{d}\mathbf{x}\\
		&=\left(a+1\right)^{2}\left(b+1\right)c-a^{2}bc+\left(a+1\right)^{2}\left(b+1\right)\left(c+1\right)-\left(a+1\right)^{2}\left(b+1\right)c\\
		&=\left(a+1\right)^{2}\left(b+1\right)\left(c+1\right)-a^{2}bc
	\end{align*}
	
	\section*{1.3}
	$\mathbf{v}=\left(\begin{array}{cc}
		y-x\\3x+2y
	\end{array}\right),D:=\left\{\left(\begin{array}{cc}
	    x\\y
	\end{array}\right)\in\mathbb{R}^{2}\middle|0\leq x,y\leq1,y\leq x\right\}$
	\begin{align*}
		\ds\int_{C}\left(y-x\right)\mathrm{d}x+\left(3x+2y\right)\mathrm{d}y&=\ds\iint_{D}\nabla\times\left(\begin{array}{cc}
			y-x\\3x+2y
		\end{array}\right)\mathrm{d}x\mathrm{d}y\\
		&=\ds\iint_{D}2\mathrm{d}x\mathrm{d}y\\
		&=\ds\int_{0}^{1}\ds\int_{0}^{x}2\mathrm{d}y\mathrm{d}x\\
		&=\ds\int_{0}^{1}2x\mathrm{d}x\\
		&=1
	\end{align*}
	
	\section*{1.4}
	\subsection*{(i)}
	\begin{align*}
		\dfrac{\partial}{\partial x}\left(\dfrac{x}{x^{2}+y^{2}}\right)-\dfrac{\partial}{\partial y}\left(\dfrac{-y}{x^{2}+y^{2}}\right)
		&=\dfrac{-x^{2}+y^{2}}{\left(x^{2}+y^{2}\right)^{2}}-\dfrac{-x^{2}+y^{2}}{\left(x^{2}+y^{2}\right)^{2}}\\
		&=0
	\end{align*}
	
	\subsection*{(ii)}
	場合分けにして、$\left(\begin{array}{cc}
		0\\0
	\end{array}\right)$は閉曲線の内部に存在しないとき
	\begin{align*}
		\ds\int_{C}\dfrac{-y}{x^{2}+y^{2}}\mathrm{d}x+\dfrac{x}{x^{2}+y^{2}}\mathrm{d}y
		&=\ds\iint_{D}0\mathrm{d}x\mathrm{d}y\\
		&=0
	\end{align*}
	$\left(\begin{array}{cc}
		0\\0
	\end{array}\right)$が内部に存在するとき、計算便利のため$C=\left(\begin{array}{ccc}
	    r\cos{\theta}\\
	    r\sin{\theta}
	\end{array}\right)$とする.$C'=\left(\begin{array}{cc}
	    -r\sin{\theta}\\
	    r\cos{\theta}
	\end{array}\right)$
	\begin{align*}
		\ds\int_{C}\mathbf{v}\mathrm{d}\mathbf{x}
		&=\ds\int_{0}^{2\pi}\left(\begin{array}{cc}
			\dfrac{-r\sin{\theta}}{r^{2}}\\
			\dfrac{r\cos{\theta}}{r^{2}}
		\end{array}\right)\cdot\left(\begin{array}{cc}
		    -r\sin{\theta}\\
		    r\cos{\theta}
		\end{array}\right)\mathrm{d}\theta\\
		&=\ds\int_{0}^{2\pi}1\mathrm{d}\theta\\
		&=2\pi
	\end{align*}
	
	
	
	
	
	
	
	
	
	
	
	
	
	
	
	
	
	
	
	
	
	
	
	
	
	
	
	
	
	
	
	
	
	
	
	
	
	
	
	
	
	
	
	
	
	
	
	
	
	
	
	
	\newpage
	\bibliographystyle{plain}
	\bibliography{references}
\end{document}
